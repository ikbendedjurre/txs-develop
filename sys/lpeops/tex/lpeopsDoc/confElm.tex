\chapter{confElm}

\section{Introduction}
One way in which information about confluence can be used before state space generation, is by appending confluent ISTEPs to other summands.
This has the effect that other branches that could follow those summands are ignored, reducing the size of the state space while maintaining its equivalence up to branching bisimulation.

\section{Algorithm}

Consider all possible pairs of summands of the LPE (including symmetric pairs).
Of a given summand pair $(s, t)$, let $t$ be a \emph{definite successor} of $s$ if $s$ contains a recursive process instantiation and if the following equation is unsatisfiable:
\begin{align*}
\neg c_s \lor \neg {c_t}[p \rightarrow v_s(p) \;|\; p \in P]
\end{align*}

where

\begin{itemize}
\item $c_s$ and $c_t$ are the guards of summands $s$ and $t$, respectively;
\item $P$ is the set of all process parameters;
\item $v_s$ is a function that yields the expression that summand $s$ assigns to a given process parameter in its recursive process instantiation.
\end{itemize}

For each summand $s$, let $D_s$ be the set of all definite successors of $s$.
$D_s$ is an underapproximation of the actual successors of $s$.

\clearpage
The algorithm follows these steps:

\begin{enumerate}

\item Determine which ISTEP summands of the LPE are confluent (see Chapter~\ref{confcheck}).

\item For each summand $s$, determine $D_s$.
Check if $D_s$ contains a confluent ISTEP summand $t$.
If so, let

\begin{lstlisting}[mathescape]
$s$ = $A$ ? $a_1$ ? $\cdots{}$ ? $a_m$ [[$c_s$]] >-> P($v_s(x_1)$, $\cdots{}$, $v_s(x_n)$)
$t$ = ISTEP [[$c_t$]] >-> P($v_t(x_1)$, $\cdots{}$, $v_t(x_n)$)
$\lor$
$t$ = ISTEP [[$c_t$]] >-> STOP
\end{lstlisting}

where

\begin{itemize}
\item $x_1$ to $x_n$ are all process parameters of the LPE;
\item $v_s$ and $v_t$ are functions that yield the expressions that summands $s$ and $t$, respectively, assign to a given process parameter in their recursive process instantiation.
\end{itemize}

Furthermore, let $\rho_s = [x_1 \rightarrow v_s(x_1), \cdots{}, x_n \rightarrow v_s(x_n)]$.

Replace $s$ by $s'$:

\begin{lstlisting}[mathescape]
$s'$ = $A$ ? $a_1$ ? $\cdots{}$ ? $a_m$ [[$c_s$]] >-> P($v_t(x_1)[\rho_s]$, $\cdots{}$, $v_t(x_n)[\rho_s]$)
\end{lstlisting}

\end{enumerate}

\clearpage
\section{Example}

Consider the following example:

\begin{lstlisting}
//Process definition:
PROCDEF example[A :: Int](x, y :: Int)
  = A >-> example[A]((x+1) mod 3, y)
  + ISTEP >-> example[A](x, (y+1) mod 4)
  ;

//Initialization:
example[A](0, 0);
\end{lstlisting}

The second summand is a confluent ISTEP summand.
It is also a definite successor of the first summand.
This means that the second summand can be appended to the first summand.

The second summand is also a definite successor of itself.
This means that the second summand can also be appended to itself.

Therefore, the original process can be changed to

\begin{lstlisting}
//Process definition:
PROCDEF example[A :: Int](x, y :: Int)
  = A >-> example[A]((x+1) mod 3, (y+1) mod 4)
  + ISTEP >-> example[A](x, (((y+1) mod 4)+1) mod 4)
  ;

//Initialization:
example[A](0, 0);
\end{lstlisting}


