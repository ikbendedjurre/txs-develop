\chapter{confcheck} \label{confcheck}

\section{Introduction}

Confluence checking is the analysis of an LPE to find ISTEP summands that are \emph{confluent}.
A confluent ISTEP summand is a summand with the property that the possible behavior of a system is the same up to branching bisimulation before and after applying that summand.

If branching bisimulation is sufficient in terms of accuracy, confluence information can be used to speed up state space generation by prioritizing confluent ISTEP summands.
Note that this means that confluence checking does not bring benefits on its own!

\section{Algorithm}

Consider all possible pairs of summands $(s, t)$ of the LPE where $s$ is a summand that communicates exclusively over the channel ISTEP and does a recursive process instantiation.
Then

\begin{lstlisting}[mathescape]
$s$ = ISTEP [[$c_s$]] >-> P($v_s(x_1)$, $\cdots{}$, $v_s(x_n)$)
$t$ = $A$ ? $a_1$ ? $\cdots{}$ ? $a_m$ [[$c_t \land a_1 = h_1 \land \cdots{} \land a_m = h_m$]]
                      >-> P($v_t(x_1)$, $\cdots{}$, $v_t(x_n)$)
\end{lstlisting}

where

\begin{itemize}
\item $x_1$ to $x_n$ are all process parameters of the LPE;
\item $v_s$ and $v_t$ are functions that yield the expressions that summands $s$ and $t$, respectively, assign to a given process parameter in their recursive process instantiation.
\end{itemize}

Furthermore, let
\begin{align*}
\rho_s &= [x_1 \rightarrow v_s(x_1), \cdots{}, x_n \rightarrow v_s(x_n)] \\
\rho_t &= [x_1 \rightarrow v_t(x_1), \cdots{}, x_n \rightarrow v_t(x_n)]
\end{align*}

If for a particular ISTEP summand $s$ the following condition holds for all pairs $(s, t)$ such that $s \neq t$, then $s$ is confluent:
\begin{align*}
c_s \land c_t \rightarrow{} &c_s[\rho_t] \land c_t[\rho_s] \\
&{} \land h_1 = h_1[\rho_s] \land \cdots{} \land h_m = h_m[\rho_s] \\
&{} \land x_1[\rho_s][\rho_t] = x_1[\rho_t][\rho_s] \land \cdots{} \land x_n[\rho_t][\rho_s] = x_n[\rho_s][\rho_t]
\end{align*}

\section{Example TOCHECK}

Consider the following example:

\begin{lstlisting}
//Process definition:
PROCDEF example[A :: Int](x, y :: Int)
  = A ? i [[x<=9 && x==i]] >-> example[A](x+1, y)
  + ISTEP [[y<=9]] >-> example[A](x, y+1)
  ;

//Initialization:
example[A](0, 0);
\end{lstlisting}

Is the second summand confluent?

If $c_s \land c_t$ holds, then $c_s[\rho_t] \land c_t[\rho_s]$ holds as well.
Since $h_i$ does not use $y$, it is unaffected by $\rho_s$.
Finally,
\begin{align*}
x[\rho_s][\rho_t] = x[\rho_t][\rho_s] = x+1 \\
y[\rho_s][\rho_t] = y[\rho_t][\rho_s] = y+1
\end{align*}

and therefore the confluence condition holds.

So yes, the second summand is confluent.

To store the new information about the second summand, the channel is renamed to CISTEP:

\begin{lstlisting}
//Process definition:
PROCDEF example[A :: Int](x, y :: Int)
  = A ? i [[x<=9 && x==i]] >-> example[A](x+1, y)
  + CISTEP [[y<=9]] >-> example[A](x, y+1)
  ;

//Initialization:
example[A](0, 0);
\end{lstlisting}

