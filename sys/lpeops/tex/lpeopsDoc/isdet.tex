\chapter{isdet}

\section{Introduction}

The \texttt{isdet} command checks whether the specified LPE is deterministic.
The command may yield false negatives.

\section{Background theory}

\subsection{Summand non-determinism}

Consider two summands, $s_1$ and $s_2$, defined as
\begin{align*}
s_i = C_i \; \texttt{?} \; x_i(1) \; \cdots{} \; \texttt{?} \; x_i(m_i) \; [[g_i]] \; \texttt{>->} \; P(v_i(p_1), \cdots{}, v_i(p_k))
\end{align*}

where

\begin{itemize}
\item $C_i$ is the name of the channel over which summand $s_i$ communicates;
\item $m_i \geq 0$ is the number of variables that summand $s_i$ uses to communicate over channel $C_i$;
\item $x_i(j)$ is the variable that summand $s_i$ uses to communicate the $j$th value of channel $C_i$;
\item $g_i$ is the guard of summand $s_i$ (the only free variables in this expression must be LPE parameters or communication variables of summand $s_i$);
\item $P$ is the LPE to which summand $s_i$ belongs;
\item $p_1, \cdots{}, p_k$ are the parameters of the LPE, of which there are $k \geq 0$;
\item $v_i(p)$ is an expression that defines the new value of LPE parameter $p$ after the application of summand $s_i$ (the only free variables in this expression must be LPE parameters or communication variables of summand $s_i$).
\end{itemize}

Summand $s_1$ is said to be \emph{deterministic with} $s_2$ if one of these conditions holds:

\begin{itemize}
\item $s_1$ and $s_2$ communicate over different channels:
\begin{align*}
C_1 \neq C_2 \land (C_1 \in \{\istep{}, \cistep{}\} \not\leftrightarrow C_2 \in \{\istep{}, \cistep{}\})
\end{align*}

\item $s_1$ and $s_2$ are never simultaneously enabled.
This is the case if the following expression has definitely no solution:
\begin{align*}
g_1 \land g_2[ x_2(j) \rightarrow q(x_2(j)) \;|\; 1 \leq j \leq m_2 ]
\end{align*}

where $q(x)$ is a surjective function that yields fresh variables.
\end{itemize}

Note that this approach is \emph{not} guaranteed to correctly recognize that two summands are deterministic (false negatives are tolerated)!

\section{Implementation}

The algorithm invoked by the \texttt{isdet} command checks for all pairs of different summands whether the first summand is deterministic with the second summand (see the previous section).
The algorithm yields \textbf{true} if and only if this is the case for all summand pairs.

%\begin{lstlisting}
%//Process definition:
%PROCDEF counterexample[A :: Int](x :: Int)
  %= A ? i [[x mod 2 == 0 && i mod 2 == 0]] >-> example[A](i + 1)
  %+ A ? i [[x == 11 && i == 11]] >-> example[A](11)
  %+ A ? i [[x == 13 && i == 13]] >-> example[A](13)
  %+ A ? i [[x == 17 && i == 17]] >-> example[A](17)
  %;
%
%//Initialization:
%example[A, B](0, 0);
%\end{lstlisting}

\section{Example TODO}

Consider the following LPE:

\begin{lstlisting}
//Process definition:
PROCDEF example[A :: Int, B](x, y :: Int)
  = A ? i [[x==0]] >-> example[A, B](1, i)
  + A ? i [[x==1 && i==y]] >-> example[A, B](2, y)
  + B [[x==2]] >-> example[A, B](3, y)
  + B [[x==3]] >-> example[A, B](0, y)
  ;

//Initialization:
example[A, B](0, 0);
\end{lstlisting}

Finding the successors of each summand is easy: each summand has exactly one successor, namely the next one, except in case of the fourth summand, where the first summand is the successor.

It is also obvious that $x$ will always be in $R_s$ for each summand $s$, because each summand uses $x$ in its guard.

Process parameter $y$ will always be in $R_{s_1}$, where $s_1$ is the first summand, because $y$ is used in the guard of $s_1$'s successor (the second summand).
After a few iterations, however, $y$ is removed from $R_{s_2}$ to $R_{s_4}$.
This means that $y$ is assigned a default value in the corresponding summands.
Depending on the mood of the SMT solver, this could give

\begin{lstlisting}
//Process definition:
PROCDEF example[A :: Int, B](x, y :: Int)
  = A ? i [[x==0]] >-> example[A, B](1, i)
  + A ? i [[x==1 && i==y]] >-> example[A, B](2, 0)
  + B [[x==2]] >-> example[A, B](3, 0)
  + B [[x==3]] >-> example[A, B](0, 0)
  ;

//Initialization:
example[A, B](0, 0);
\end{lstlisting}




