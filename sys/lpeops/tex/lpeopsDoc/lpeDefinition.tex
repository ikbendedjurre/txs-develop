\chapter{LPE}

\section{Introduction}
The techniques that are described in this document require a \emph{linear process equation} (LPE) as input.
Many, but not all models can be transformed to LPE format using the \texttt{lpe} command of \txs{}.

\section{Components}

An LPE must consist of exactly the following components:

\begin{itemize}

\item The \emph{signature} of a process $P$.
This includes at least its name, its channel parameters (name and type), and its data parameters (name and type).

\item An \emph{instantiation} of process $P$.
The instantiation must match the signature of process $P$, including channel parameter values and data parameter values.
The expressions for data parameter values may \emph{not} contain any free variables!

\item The \emph{body} of process $P$.
The body consists of one or more \emph{summands}.
In this context, a summand must contain exactly the following, in the given order:

\begin{itemize}

\item Zero or more \emph{channel communications}; that is, channel references followed by the variables that are used to communicate over those channels.
Multiple channel communications are combined with \inlinecode{|}.
Channel variables may occur only once in the channel communications of a summand; they \emph{can} occur in the channel communications of multiple summands.

Examples: \inlinecode{Output}, \inlinecode{Output ? x ? y}, \inlinecode{Input ? x | Output ? y}.

\item A guard, yielding a boolean value.
The only free variables in the guard must be parameters of $P$ or communication variables of the same summand.

\item A sequence operator, \inlinecode{>->}.

\item A recursive process instantiation that matches the signature of process $P$.
This includes channel parameter values and data parameter values; the only free variables in the expressions for data parameter values must be LPE parameters or communication variables of the same summand.

\end{itemize}

\end{itemize}

\section{Example TODO}

The following code snippet gives a valid example of an LPE:

\begin{lstlisting}
//Process definition:
PROCDEF example[A :: Int, B](state, curr, prev :: Int)
  = A ? i [[state==0]] >-> example[A, B](2, i, prev)
  + A ? i [[state==1 && i!=prev]] >-> example[A, B](2, i, prev)
  + A ? i [[state==2 && i==curr]] >-> example[A, B](1, curr, curr)
  + B >-> STOP
  ;

//Initialization:
example[A, B](0, 0, 0);
\end{lstlisting}

The process only accepts input sequences in which every number is repeated exactly once.
The process terminates non-deterministically.
