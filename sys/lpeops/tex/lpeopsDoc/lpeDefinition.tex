\chapter{LPE}

\section{Introduction}
The techniques that are described in this document require a \emph{linear process equation} (LPE) as input.
Many, but not all models can be transformed to LPE format using the \texttt{lpe} command of \txs{}.

\section{Requirements}

An LPE must consist of exactly the following elements:

\begin{itemize}

\item The \emph{signature} of a process $P$.
This includes at least its name, its channel parameters (name and type), and its data parameters (name and type).

\item An \emph{instantiation} of process $P$.
The instantiation must be a closed expression within the context of the signature of process $P$ and a sufficient number of channels to provide channel parameter values.
Otherwise, the process instantiation must satisfy the usual requirements of \txs{}' process instantiations.
There are also no requirements for the channel parameters and data parameters beyond the usual requirements of \txs{}' process definitions.

\item The \emph{definition} of process $P$ consists of one or more \emph{summands}.
In this context, a summand must contain exactly the following, in the given order:

\begin{itemize}

\item One or more \emph{channel communications}; that is, channel references followed by the variables that are used to communicate over those channels.
For example, \inlinecode{OUTPUT ? x ? y}, or simply \inlinecode{OUTPUT}.

Channel communications are combined with \inlinecode{|}.
To give another example, \inlinecode{INPUT ? x | OUTPUT ? y}.

The variables that are used to communicate over a channel must be \emph{fresh}.

\item A guard, yielding a boolean value.
The guard is not always explicitly written if it is semantically equivalent to \inlinecode{true}.

\item A sequence operator, \inlinecode{>->}.

\item The expression for deadlock (\inlinecode{STOP}) \emph{or} a recursive process instantiation.
The instantiated process must use the process definition given by the LPE, which satisfies the requirements of a \txs{} process signature.
This includes channel parameters: these must be assigned their current values of the instantiating process.
There are no requirements for the values assigned to the data parameters beyond the usual requirements of \txs{} (such as type compatibility).

\end{itemize}

\end{itemize}

\section{Example}

The following code snippet gives a valid example of an LPE:

\begin{lstlisting}
//Process definition:
PROCDEF example[A :: Int, B](state, curr, prev :: Int)
  = A ? i [[state==0]] >-> example[A, B](2, i, prev)
  + A ? i [[state==1 && i!=prev]] >-> example[A, B](2, i, prev)
  + A ? i [[state==2 && i==curr]] >-> example[A, B](1, curr, curr)
  + B >-> STOP
  ;

//Initialization:
example[A, B](0, 0, 0);
\end{lstlisting}

The process only accepts input sequences in which every number is repeated exactly once.
The process terminates non-deterministically.
