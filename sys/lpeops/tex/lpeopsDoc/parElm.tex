\chapter{parElm}

\section{Introduction}

A linearized process may have parameters that do not affect the behavior of that process in any way.
These parameters are called \emph{inert}.

Removing inert parameters from a process has the advantages of fewer states, smaller state vectors (and therefore reduced memory usage), and faster performance in general.
These advantages may be small; however, detection and removal of inert parameters can be done fairly cheaply.

\section{Algorithm}

The algorithm consists of the following steps \cite{groote2001computer}:

\begin{enumerate}

\item Mark all process parameters.
(This means that, initially, we assume that all process parameters are inert.)

\item Unmark all process parameters that occur in the guard of a summand.

\item Consider the assignments that occur as part of the recursive process instantiations of the summands of the LPE.
Unmark all process parameters that occur in the expression of which the value is assigned to an \emph{unmarked} process parameter.

\item Repeat the previous step until no process parameter is unmarked.
All remaining marked process parameters can be safely removed the process.

\end{enumerate}

\clearpage
\section{Example}

Consider the following LPE:

\begin{lstlisting}
//Process definition:
PROCDEF example[A :: Int, B](x, y, z :: Int)
  = A ? i [[x==0]] >-> example[A, B](i, y, z)
  + A ? i [[x==1]] >-> example[A, B](0, i, z)
  + B [[x==2]] >-> example[A, B](0, y, z)
  + B >-> example[A, B](z, y, x)
  ;

//Initialization:
example[A, B](0, 0, 0);
\end{lstlisting}

First, $x$ is unmarked, because it occurs in the guards of the first three summands.

In the fourth summand, $z$ is used in the expression of which the value is assigned to $x$.
Therefore $z$ must also be unmarked.

$y$ remains marked: it does not occur in a guard, nor is it used in the assignment to a process parameter other than itself.
Removing $y$ gives

\begin{lstlisting}
//Process definition:
PROCDEF example[A :: Int, B](x, z :: Int)
  = A ? i [[x==0]] >-> example[A, B](i, z)
  + A ? i [[x==1]] >-> example[A, B](0, z)
  + B [[x==2]] >-> example[A, B](0, z)
  + B >-> example[A, B](z, x)
  ;

//Initialization:
example[A, B](0, 0);
\end{lstlisting}



