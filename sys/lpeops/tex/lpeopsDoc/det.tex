\chapter{det}

\section{Introduction}

Non-determinism in an LPE can have a strong impact on the performance of exploring its state space.
Preferably, an LPE should therefore be completely deterministic.
Unfortunately, there are no determinization methods known for symbolic models such as LPEs.

The \texttt{det} command described in this section is only a naive attempt at LPE determinization: it looks for two summands that are non-deterministic (see \ref{isdet}), and then rewrites the LPE so that these summands are deterministic.
This may mean that the entire LPE has become deterministic, but more typically the non-determinism will have moved to a different part of the LPE.

The \texttt{det} command is intended to be repeated in the hope that non-determinism eventually disappears from the LPE.
This may never happen!

It is advised to symbolically reduce the LPE using other LPE operations after each application of \texttt{det}.

\section{Algorithm}

Consider an LPE of which two summands, $s_1$ and $s_2$, are non-deterministic (see \ref{isdet}).
Define $s_1$ and $s_2$ such that
\begin{align*}
s_i = C \; \texttt{?} \; x_i(1) \; \cdots{} \; \texttt{?} \; x_i(m) \; [[g_i]] \text{ \texttt{>->} } P(v_i(p_1), \cdots{}, v_i(p_k))
\end{align*}

where

\begin{itemize}
\item $C$ is the name of the channel over which summands $s_1$ and $s_2$ communicate;
\item $m \geq 0$ is the number of variables that summands $s_1$ and $s_2$ use to communicate over channel $C$;
\item $x_i(j)$ is the variable that summand $s_i$ uses to communicate the $j$th value of channel $C$;
\item $g_i$ is the guard of summand $s_i$ (the only free variables in this expression must be LPE parameters or communication variables of summand $s_i$);
\item $P$ is the LPE to which summand $s_i$ belongs;
\item $p_1, \cdots{}, p_k$ are the parameters of the LPE, of which there are $k \geq 0$;
\item $v_i(p)$ is an expression that defines the new value of LPE parameter $p$ after the application of summand $s_i$ (the only free variables in this expression must be LPE parameters or communication variables of summand $s_i$);
\end{itemize}

\begin{enumerate}
\item Remove $s_1$ and $s_2$ from the LPE.

\item Change the guard of each remaining summand in the LPE from $g$ to $\neg f \land g$.

\item Add two new summands ${s_1}'$ and ${s_2}'$ to the LPE, which are defined as
\begin{align*}
{s_1}' = C \; \texttt{?} \; x_1(1) \; &\cdots{} \; \texttt{?} \; x_1(m) \; [[\neg f \land g_1 \land \neg g_2[X_1]]] \\
&\texttt{>->} \; P(v_1(p_1), \cdots{}, v_1(p_k), \textbf{false}, []) \\
{s_2}' = C \; \texttt{?} \; x_2(1) \; &\cdots{} \; \texttt{?} \; x_2(m) \; [[\neg f \land \neg g_1[X_2] \land g_2]] \\
&\texttt{>->} \; P(v_2(p_1), \cdots{}, v_2(p_k), \textbf{false}, [])
\end{align*}

\item Add a new summand $s_3$ to the LPE, which is defined as
\begin{align*}
s_3 = C \; \texttt{?} \; x_3(1) \; &\cdots{} \; \texttt{?} \; x_3(m) \; [[\neg f \land g_1[X_3] \land g_2[X_3]]] \\
&\texttt{>->} \; P(p_1, \cdots{}, p_k, \textbf{true}, [x_3(1), \cdots{}, x_3(m)])
\end{align*}

\item For each possible successor $t$ of $s_i$, add the following summand to the LPE:
\begin{align*}
t = C_t \; \texttt{?} \; x_t(1) \; &\cdots{} \; \texttt{?} \; x_t(m_t) \; [[f \land g_t[V_i]]] \\
&\texttt{>->} \; P(v_t(p_1)[V_i], \cdots{}, v_t(p_k)[V_i], \textbf{false}, [])
\end{align*}

\end{enumerate}

\section{Example}

Consider the following LPE:

\begin{lstlisting}
//Process definition:
PROCDEF example[A :: Int, B](x, y :: Int)
  = A ? i [[x==0]] >-> example[A, B](1, i)
  + A ? i [[x==1 && i==y]] >-> example[A, B](2, y)
  + B [[x==2]] >-> example[A, B](3, y)
  + B [[x==3]] >-> example[A, B](0, y)
  ;

//Initialization:
example[A, B](0, 0);
\end{lstlisting}

TODO

