\chapter{uguard}

\section{Introduction}

TODO

\section{Formal background}

\subsection{Summand implication}

Consider two summands, $s_\alpha$ and $s_\beta$, and reference their elements conform \ref{summandelements}.

Summand $s_\alpha$ is said to \emph{imply} $s_\beta$ if these two conditions hold:

\begin{itemize}
\item $s_\alpha$ and $s_\beta$ must communicate over the exact same channel with exactly as many channel variables; that is, $C_\alpha = C_\beta \land m_\alpha = m_\beta$.

\item Given the previous condition, define $m = m_\alpha = m_\beta$ and, in particular, the mapping
\begin{align*}
X_\beta = [x_\alpha(j) \rightarrow x_\beta(j) \;|\; 1 \leq j \leq m]
\end{align*}

The following condition must be a tautology:
\begin{align*}
g_\alpha[X_\beta] \rightarrow g_\beta
\end{align*}
\end{itemize}

\subsection{Possible successors}

Consider all possible pairs of summands $(s_\alpha, s_\beta)$ of the LPE (including symmetric pairs), and reference the elements of $s_\alpha$ and $s_\beta$ conform \ref{summandelements}.
Summand $s_\beta$ is said to be a \emph{possible successor} of $s_\alpha$ if the following expression is satisfiable:
\begin{align*}
g_\alpha \land {g_\beta}[p \rightarrow v_\alpha(p) \;|\; p \in P]
\end{align*}

\subsection{Definite successors}

Consider all possible pairs of summands $(s_\alpha, s_\beta)$ of the LPE (including symmetric pairs), and reference the elements of $s_\alpha$ and $s_\beta$ conform \ref{summandelements}.
Summand $s_\beta$ is said to be a \emph{possible successor} of $s_\alpha$ if the following expression is satisfiable:
\begin{align*}
g_\alpha \land {g_\beta}[p \rightarrow v_\alpha(p) \;|\; p \in P]
\end{align*}

\section{Algorithm}

Let the input LPE model be $M$ and the underlying LPE be $P$.
The \texttt{uguard} algorithm follows these steps when applied to $M$:

\begin{enumerate}
\item Find two summands $s_\alpha$ and $s_\beta$ such that $s_\alpha$ implies $s_\beta$.
\item Determine $D_\alpha$, the possible successors of $s_\alpha$.
\item Determine $D_\beta$, the definite successors of $s_\beta$.
\item For each summand $s_\delta \in D_\beta$ such that $s_\delta \notin D_\alpha$, do the following:
\begin{enumerate}
\item Create a new LPE $P'$ with the same parameters as $P$, but without any summands.
Create a new LPE model $M'$ that has the same definition as $M$, except that its references to $P$ have been replaced by references to $P'$.

\item Add a new, fresh parameter $f$ of type \texttt{Bool} to $P'$.
Where $M'$ instantiates $P'$, initialize $f$ with $\textbf{true}$.

\item Consider each summand $s_i \notin D_\beta$ of $P$, and reference its elements conform \ref{summandelements}.
For each summand $s_i$, add to $P'$ a new summand ${s_i}'$ that is defined as
\begin{align*}
{s_i}' = C_i \; \texttt{?} \; x_i(1) \; &\cdots{} \; \texttt{?} \; x_i(m_i) \; [[g_i]] \\
&\texttt{>->} \; P(v_i(p_1), \cdots{}, v_i(p_k), \Gamma(s_i))
\end{align*}

where
\begin{align*}
\Gamma(s) = \begin{cases}
\textbf{true} \text{ if } s \neq s_\alpha \land s \neq s_\beta \\
\textbf{false} \text{ if } s = s_\alpha \lor s = s_\beta
\end{cases}
\end{align*}

\item Consider each summand $s_i \in D_\beta$ of $P$, and reference its elements conform \ref{summandelements}.
For each summand $s_i$, add to $P'$ a new summand ${s_i}'$ that is defined as
\begin{align*}
{s_i}' = C_i \; \texttt{?} \; x_i(1) \; &\cdots{} \; \texttt{?} \; x_i(m_i) \; [[f \land g_i]] \\
&\texttt{>->} \; P(v_i(p_1), \cdots{}, v_i(p_k), \Gamma(s_i))
\end{align*}

with the same $\Gamma$ function as before.
\end{enumerate}
\end{enumerate}

\section{Example}

Consider the following LPE:

\begin{lstlisting}
//Process definition:
PROCDEF example[A :: Int, B](x, y :: Int)
  = A ? i [[x==0]] >-> example[A, B](1, i)
  + A ? i [[x==1 && i==y]] >-> example[A, B](2, y)
  + B [[x==2]] >-> example[A, B](3, y)
  + B [[x==3]] >-> example[A, B](0, y)
  ;

//Initialization:
example[A, B](0, 0);
\end{lstlisting}

TODO

