\chapter{LPE operations}

\section{Usage}

The \texttt{lpeop} command has 3 space-separated arguments. In order, these arguments are:

\begin{itemize}
\item A chain of LPE operations.
LPE operations are represented by their name, such as \texttt{cstelm} or \texttt{loop}.
These names are separated by the symbol \texttt{->}.

LPE operations in the chain are executed from left to right, passing their output to the next LPE operation as input.
If a problem occurs, the process ends immediately.
Otherwise, the final LPE is saved as a new model.
\item The name of the input model.
The input model should be in the LPE format (obviously).
\item A base name \textit{base} for generated output.
The name is primarily used for the output model (if any), which is in the LPE format.
However, LPE operations may adopt the name for their own purposes.
The \texttt{export} operation, for example, will create a file by that name with the \texttt{.txs} extension.

It is possible to use the \texttt{\%i} token in the base name.
This will insert the current counter value into the name.
Use the \texttt{inc} command to increase the counter (which starts at 1).
\end{itemize}

\section{Basic operations}

\begin{tabularx}{\linewidth}{l|X|}
\textbf{Operation} & \textbf{Description} \\ \hline
\texttt{stop} & End the chain of operations immediately. \\ \hline
\texttt{show} & Print the input LPE to the terminal. \\ \hline
\texttt{show*} & Same as \texttt{show}, but the output is compilable \txs{} code, and identifiers in the output have been shortened for readability. \\ \hline
\texttt{export} & Save the input LPE to \textit{base}\texttt{.txt}. \\ \hline
\texttt{export*} & Same as \texttt{export}, but the output is compilable \txs{} code, and identifiers in the output have been shortened for readability. \\ \hline
\texttt{loop} & If the input LPE has not been encountered before, restart from the most recent \texttt{start} command, or from the first operation if no \texttt{start} command has been encountered yet. \\ \hline
\texttt{loop*}$N$ & Same as \texttt{loop}, but the number of restarts is limited to $N$. \\ \hline
\texttt{start} & Set the location where the command chain should continue when looping. \\ \hline
\texttt{inc} & Increase the counter that can be used in \textit{base}. \\ \hline
\end{tabularx}

\section{Analysis and rewrite methods}

\begin{tabularx}{\linewidth}{l|X|}
\textbf{Operation} & \textbf{Description} \\ \hline
\texttt{clean} & Remove summands of which it can be established that there already exists an equivalent instance in the LPE from the input LPE; and remove summands of which it can be established via symbolic reachability that they cannot be reached from the input LPE. \\ \hline
\texttt{cstelm} & Remove parameters of which it can be established that their value never changes from the input LPE. \\ \hline
\texttt{parelm} & Remove parameters of which it can be established that their value is never used from the input LPE. \\ \hline
\texttt{parreset} & Set parameters of which it can be established via symbolic reachability that their value is no longer used after a specific summand to a default value in the process instantiation of that summand. \\ \hline
\texttt{datareset} & Set parameters of which it can be established via control-flow analysis that their value is no longer used after a specific summand to a default value in the process instantiation of that summand. \\ \hline
\texttt{confcheck} & Returns the input LPE after assessing whether it contains confluent \texttt{ISTEP} summands. \\ \hline
\texttt{confelm} & Rewrite the input LPE so that confluent \texttt{ISTEP} summands are prioritized. \\ \hline
\texttt{istepelm} & Rewrite the input LPE so that \texttt{ISTEP} channels no longer occur. \\ \hline
\texttt{mcrl2} & Convert the current LPE to an \mcrl{} specification, and save it to \textit{base}\texttt{.mcrl2}. \\ \hline
\texttt{isdet} & Return the input LPE after assessing whether it is deterministic. May yield false negatives. \\ \hline
\texttt{det} & \textit{Experimental.} Rewrite the input LPE in an attempt to reduce non-determinism. By design, \texttt{det} does not generally remove all non-determinism in one execution because it may not terminate (so use with \texttt{loop}). \\ \hline
\texttt{uguard} & \textit{Experimental.} Search for underspecified summands and remove them from the input LPE. \\ \hline
\end{tabularx}

\section{Operation properties}

\begin{tabularx}{\linewidth}{X|c|c|c|c|}
\textbf{Operation} & \textbf{u-ioco} & \textbf{weak bis.} & \textbf{strong bis.} & \textbf{state sp. equiv.} \\ \hline
\texttt{clean} & Yes & Yes & Yes & Yes \\ \hline
\texttt{cstelm} & Yes & Yes & Yes & Yes\footnote{\label{statevectors}Only change is that state vectors may be smaller.} \\ \hline
\texttt{parelm} & Yes & Yes & Yes & Yes\footnoteref{statevectors} \\ \hline
\texttt{parreset} & Yes & Yes & Yes & No \\ \hline
\texttt{datareset} & Yes & Yes & Yes & No \\ \hline
\texttt{confcheck} & Yes & Yes & Yes & Yes \\ \hline
\texttt{confelm} & Yes & Yes & No & No \\ \hline
\texttt{istepelm} & Yes & Yes & No & No \\ \hline
\texttt{mcrl2} & Yes & Yes & Yes & Yes \\ \hline
\texttt{isdet} & Yes & Yes & Yes & Yes \\ \hline
\texttt{det} & Yes & Yes & Yes & Yes\footnote{Only change is that state vectors may be larger.} \\ \hline
\texttt{uguard} & Yes & No & No & No \\ \hline
\end{tabularx}

