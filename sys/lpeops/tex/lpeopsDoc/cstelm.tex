\chapter{cstelm}

\section{Introduction}

An LPE may have parameters that do not change value throughout the state space of that process.
These parameters are essentially \emph{constant}.
Removing constant parameters results in a smaller state vector (which may improve performance), but otherwise the state space of an LPE remains the same.

Constants can be safely removed from a process as follows:
\begin{itemize}
\item Substitute references to a constant parameter by the initial (explicit) value of that parameter;
\item Remove the constant from the process parameter list.
\end{itemize}

\section{Implementation}

The algorithm consists of the following steps \cite{groote2001computer}:

\begin{enumerate}

\item Label all LPE parameters with `constant'.

\item Let $P$ be the set of all LPE parameters that are labeled with `constant'.
Define a substitution $\rho = [p \rightarrow v_I(p) \;|\; p \in P]$, where $v_I(p)$ gives the initial value of LPE parameter $p$.

\item Consider each summand $s$ of the LPE.
Construct an equation $g_s \rightarrow p = v_s(p)$ for all $p \in P$, where $g_s$ is the guard of $s$ and where $v_s(p)$ is the expression that defines the value of LPE parameter $p$ after the application of summand $s$, and apply the substitution $\rho$ to it.
(This gives $(g_s \rightarrow p = v_s(p))[\rho] \Leftrightarrow {g_s}[\rho] \rightarrow v_I(p) = v_s(p)[\rho]$.)

If the obtained equation is a tautology (that is, if its negation is unsatisfiable) for all $s$, $p$ remains labeled with `constant'; otherwise, remove the `constant' label from $p$.

\item Repeat the previous two steps until the fixpoint of $P$ is reached.
Then remove all LPE parameters that are still labeled as `constant' from the LPE, substituting references to those parameters by their initial values.

\end{enumerate}

\section{Example TOCHECK}

Consider the following LPE:

\begin{lstlisting}
//Process definition:
PROCDEF example[A](x, y, z :: Int)
  = A [[z = 2]] >-> example[A](z-1, 1, 2)
  + A >-> example[A](y, x, x+y)
  + A >-> example[A](1, x, z+1)
  ;

//Initialization:
example[A](1, 1, 2);
\end{lstlisting}

First, $\rho = [ x \rightarrow 1, y \rightarrow 1, z \rightarrow 2 ]$.

We must check the following equations:

\begin{align*}
(z = 2 \rightarrow x = z-1)[\rho] &\Leftrightarrow (2 = 2 \rightarrow 1 = 2-1) \Leftrightarrow \textit{true} \\
(z = 2 \rightarrow y = 1)[\rho] &\Leftrightarrow (2 = 2 \rightarrow 1 = 1) \Leftrightarrow \textit{true} \\
(z = 2 \rightarrow z = 2)[\rho] &\Leftrightarrow (2 = 2 \rightarrow 2 = 2) \Leftrightarrow \textit{true} \\
(x = y)[\rho] &\Leftrightarrow (1 = 1) \Leftrightarrow \textit{true} \\
(y = x)[\rho] &\Leftrightarrow (1 = 1) \Leftrightarrow \textit{true} \\
(z = x+y)[\rho] &\Leftrightarrow (2 = 1+1) \Leftrightarrow \textit{true} \\
(x = 1)[\rho] &\Leftrightarrow (1 = 1) \Leftrightarrow \textit{true} \\
(y = x)[\rho] &\Leftrightarrow (1 = 1) \Leftrightarrow \textit{true} \\
(z = z+1)[\rho] &\Leftrightarrow (2 = 2+1) \Leftrightarrow \textit{false} \\
\end{align*}

The last equation is not a tautology, and so $z$ is unmarked.

The new value of $\rho$ is $[ x \rightarrow 1, y \rightarrow 1 ]$.

\clearpage
The equations are now the following:

\begin{align*}
(z = 2 \rightarrow x = z-1)[\rho] &\Leftrightarrow (z = 2 \rightarrow 1 = z-1) \Leftrightarrow \textit{true} \\
(z = 2 \rightarrow y = 1)[\rho] &\Leftrightarrow (z = 2 \rightarrow 1 = 1) \Leftrightarrow \textit{true} \\
(x = y)[\rho] &\Leftrightarrow (1 = 1) \Leftrightarrow \textit{true} \\
(y = x)[\rho] &\Leftrightarrow (1 = 1) \Leftrightarrow \textit{true} \\
(x = 1)[\rho] &\Leftrightarrow (1 = 1) \Leftrightarrow \textit{true} \\
(y = x)[\rho] &\Leftrightarrow (1 = 1) \Leftrightarrow \textit{true} \\
\end{align*}

All of the equations above are tautologies, and so $x$ and $y$ remain marked.
Removing the marked parameters from the LPE gives

\begin{lstlisting}
//Process definition:
PROCDEF example[A](z :: Int)
  = A [[z==2]] >-> example[A](2)
  + A >-> example[A](2)
  + A >-> example[A](z+1)
  ;

//Initialization:
example[A](2);
\end{lstlisting}

Obviously, more simplification is possible.

